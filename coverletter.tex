%!TEX TS-program = xelatex
%!TEX encoding = UTF-8 Unicode
% Awesome CV LaTeX Template for Cover Letter
%
% This template has been downloaded from:
% https://github.com/posquit0/Awesome-CV
%
% Authors:
% Claud D. Park <posquit0.bj@gmail.com>
% Lars Richter <mail@ayeks.de>
%
% Template license:
% CC BY-SA 4.0 (https://creativecommons.org/licenses/by-sa/4.0/)
%


%-------------------------------------------------------------------------------
% CONFIGURATIONS
%-------------------------------------------------------------------------------
% A4 paper size by default, use 'letterpaper' for US letter
\documentclass[11pt, a4paper]{awesome-cv}

% Configure page margins with geometry
\geometry{left=1.4cm, top=.8cm, right=1.4cm, bottom=1.8cm, footskip=.5cm}

% Specify the location of the included fonts
\fontdir[fonts/]

% Color for highlights
% Awesome Colors: awesome-emerald, awesome-skyblue, awesome-red, awesome-pink, awesome-orange
%                 awesome-nephritis, awesome-concrete, awesome-darknight
\colorlet{awesome}{awesome-red}
% Uncomment if you would like to specify your own color
% \definecolor{awesome}{HTML}{CA63A8}

% Colors for text
% Uncomment if you would like to specify your own color
% \definecolor{darktext}{HTML}{414141}
% \definecolor{text}{HTML}{333333}
% \definecolor{graytext}{HTML}{5D5D5D}
% \definecolor{lighttext}{HTML}{999999}

% Set false if you don't want to highlight section with awesome color
\setbool{acvSectionColorHighlight}{true}

% If you would like to change the social information separator from a pipe (|) to something else
\renewcommand{\acvHeaderSocialSep}{\quad\textbar\quad}


%-------------------------------------------------------------------------------
%	PERSONAL INFORMATION
%	Comment any of the lines below if they are not required
%-------------------------------------------------------------------------------
% Available options: circle|rectangle,edge/noedge,left/right
\photo[circle,noedge,left]{bryan}
\name{Bryan}{Ngo}
\position{Computer Science \& Engineering {\enskip\cdotp\enskip}Undergraduate}
\address{1758 River Oaks Circle}

\mobile{(+1) 707 - 718 - 9222}
\email{bryanngo97@gmail.com}
\homepage{bryngo.me}
\github{bryngo}
\linkedin{bryngo}
% \gitlab{gitlab-id}
% \stackoverflow{SO-id}{SO-name}
% \twitter{@twit}
% \skype{skype-id}
% \reddit{reddit-id}
% \extrainfo{extra informations}

% \quote{``Be the change that you want to see in the world."}


%-------------------------------------------------------------------------------
%	LETTER INFORMATION
%	All of the below lines must be filled out
%-------------------------------------------------------------------------------
% The company being applied to
\recipient
  {Github Recruitment Team}
  {Github\\88 Colin P Kelly Jr St\\San Francisco, CA 94107}
% The date on the letter, default is the date of compilation
\letterdate{\today}
% The title of the letter
\lettertitle{Job Application for Atom Software Engineer Intern}
% How the letter is opened
\letteropening{Dear Hiring Manager,}
% How the letter is closed
\letterclosing{Sincerely,}
% Any enclosures with the letter
\letterenclosure[Attached]{Resume}


%-------------------------------------------------------------------------------
\begin{document}

% Print the header with above personal informations
% Give optional argument to change alignment(C: center, L: left, R: right)
\makecvheader[R]

% Print the footer with 3 arguments(<left>, <center>, <right>)
% Leave any of these blank if they are not needed
\makecvfooter
  {\today}
  {Bryan Ngo~~~·~~~Cover Letter}
  {}

% Print the title with above letter informations
\makelettertitle

%-------------------------------------------------------------------------------
%	LETTER CONTENT
%-------------------------------------------------------------------------------
\begin{cvletter}

\lettersection{About Me}
I wrote my first line of code during my first year at UC Davis.
Prior to university, I had no idea that computer science was a subject I could get into, or even just a career I could find myself pursuing.
So, I wasn't among the fortunate few who grew up with a strong network of influences to get into the tech industry. \\

But since my first year at university, I've made it a clear goal to myself to become just as good, if not better than those who grew up around the tech industry.
I put myself around those who are smarter than me so I could learn from them and I picked up leadership positions in the computer science community; all in an effort to take on my passion to learn about the many fields computer science has to offer.
On a completely different note, I love photography and snowboarding!

\lettersection{Why Github?}
Ever since learning about version control and Github sometime during my second year of university, I've kept it as a part of my learning process.
Just like everyone else, I've used Github as a platform to host my projects, find new projects, contribute to projects, and to find community.
I could even say that I use Github like how many use social media.
I'm always checking for what cool projects my friends and community are working on / taking a look at. \\

More specifically, Atom is also part of my daily toolbox.
After using text editors such as vim and emacs for so long, Atom was a godsend.
In short, it's an amazing text editor that has so many great out-of-the-box features. \\

So in summary, it just makes sense to want to work for a company whose product I use everyday.

\lettersection{Why Me?}
In addition to being passionate about programming, I believe I add diverse set of ideas to the team.
Generally speaking, I've used all the products that enable text editing including terminal based text editors such as vim or emacs, standard text editors like Sublime and, of course, Atom, and IDEs.
Over the years, I've identified what works and what doesn't with each platform, and I want to provide this valuable insight to make the best hackable text editor even better!

\end{cvletter}


%-------------------------------------------------------------------------------
% Print the signature and enclosures with above letter informations
\makeletterclosing

\end{document}
